\documentclass[12pt,a4paper]{article}
\usepackage{times}
\usepackage{durhampaper}
\usepackage{harvard}

\citationmode{abbr}
\bibliographystyle{agsm}

\title{Reinforcement Learning: \\Transfer Learning Between Different Games}
\author{} % leave; your name goes into \student{}
\student{Matthew Chapman}
\supervisor{Dr Lawrence Mitchell}
\degree{BSc Natural Sciences}

\date{\today}

\begin{document}

\maketitle

\begin{abstract}

\begin{flushleft} 
% change to success in visually complex domains
{\bf Background} --- The recent successes of reinforcement learning systems solving complex decision-making problems means that such systems have potential for real-world applications like robotics and self-driving cars, providing advantages over other machine learning approaches. Existing solutions perform strongly when solving a certain problem but adapt poorly to solve related problems, and so research is being done to improve this.
\end{flushleft}

\begin{flushleft}
{\bf Aims} --- The aim of this project is to investigate the potential of transfer learning in reinforcement learning by assessing the benefit of pre-training agents. We aim to explore whether current state-of-the-art architectures have the capacity to adapt to new environments, and whether generative models can pre-train from multiple agents.
\end{flushleft}

\begin{flushleft}
{\bf Method} --- The method is to implement a modern reinforcement learning algorithm and train it on two selected games. Once the agent is performing well in the environment, we save a video of it playing, and use the video as the basis of pre-training of another agent, which we then evaluate on the test game.    
\end{flushleft}

\begin{flushleft}
{\bf Results} --- Pre-training leads to the agent performing better, as measured by an improvement in the following three metrics: jump-start performance, accumulated reward, and final performance. The more pre-training information the agent has, the better it performs. 
\end{flushleft}

\begin{flushleft}
{\bf Conclusions} --- Transfer learning is a promising method to prepare agents for environments where it has limited opportunities to interact with and learn from. By training agents on similar environments, we can build confidence that the agent will perform successfully when evaluated on the test environment. This is particularly useful in fields such as autonomous driving, where the vehicle must be able to adapt to various changes in the environment.
\end{flushleft}

\end{abstract}

\begin{keywords}
Artificial intelligence, machine learning, reinforcement learning, deep learning, transfer learning.
\end{keywords}

\newpage
\section{Introduction (2-3 pages)}
This project is about reinforcement learning and transfer learning. The project involves developing a reinforcement learning algorithm to learn to perform successfully in some environment. Additionally, the project involves investigating how transfer learning lets the algorithm store knowledge and apply it --- to learn to perform successfully in a different but related environment.

\subsection{Background} 
\subsubsection{Reinforcement learning}
Reinforcement learning is the class of problems concerned with an agent learning behaviour through trial-and-error interactions with a dynamic environment \cite{Kaelbling1996}. An example of a problem is an aspiring tightrope walker (the agent) learning to maintain balance (the behaviour) while walking along a tightrope that contorts and wobbles under their weight (the dynamic environment). With each attempt and fall (the trial-and-error interactions), the walker learns how better to correct their balance, and adjusts their behaviour slightly for the next attempt. When the walker is able to maintain balance consistently over consecutive attempts, the desired behaviour is achieved, and so the learning task is complete. We say that the problem is solved and the reinforcement learning agent has learned to perform successfully in the environment.  

There are algorithms that act as agents that solve reinforcement learning problems. These reinforcement learning algorithms can solve problems in physical settings, such as driving cars, or in virtual settings, such as playing games. We can treat these algorithms as functions that takes as input observations and outputs actions. Examples of observations are the video from a camera attached to a self-driving car or the positions of pieces on a chessboard in an online match. Examples of corresponding actions are to turn the steering wheel in one direction or to move a chess piece. The goal of the algorithm is to learn which actions are the best to take given some observations. How good an action given an observation is can be measured by how likely taking the action is to lead to the desired behaviour. For an algorithm that drives cars, the desired behaviour might be to drive safely, and so the algorithm would know to stop at a red traffic light. Whereas, for an algorithm that plays chess, the desired behaviour might be to win, and so the algorithm would know to take the opponent's king. The algorithm is learning a mapping, from actions and observations to values, to inform its decision-making. This mapping is initially unknown, but improves the more the algorithm interacts with its environment --- the same way one gets better with practice at driving or playing chess. For relatively complex problems, there may not be an optimal solution, such as behaviour that guarantees no accidents or that always wins, and so the best the algorithm can do is to approximate an optimal solution. 

\subsubsection{Transfer learning}
Transfer learning is the application of knowledge gained while solving one problem to solve a different but related problem \cite{2010}. An example is an agent who has learned to walk a tightrope (solving one problem) applying their balancing ability (the knowledge gained) to learn to surf (a different but related problem). By reusing knowledge gained from solving past problems, it is expected that solving a different but related problem will be more efficient than it would be without the prior knowledge. As in the example, a tightrope walker should learn to balance on a surfboard more easily and more quickly, due to their knowledge of balancing on a rope, than someone without the same acquired knowledge of balancing. 

In transfer learning for reinforcement learning algorithms, the knowledge gained that can be applied is the agent's policy. The policy is the set of rules that determine an agent's behaviour (which can be informed by the mapping mentioned in the previous section). An example of a policy for an agent playing the video game Breakout might be to move the paddle randomly. Another, better policy might be to move the paddle in the direction of the projectile, so as to deflect it. The policy of interest, however, is one that the reinforcement learning algorithm developed itself while learning to play. Depending on the architecture of the algorithm, the policy could be a neural network, so that developing the policy equates to training the network. An example of transfer learning for reinforcement learning algorithms, then, could be to apply the trained neural network, or part of it, that was trained in the agent that learned to play Breakout, to a new agent that is learning to play a different but related game, such as Pong. In both games, players must move a paddle to deflect a projectile. The hope is that an abstraction of this concept has been learned in the network, which gets translated and used to make learning other games with similar features more efficient.

\subsubsection{Context}
Reinforcement learning algorithms have proven to be successful at achieving at least human-level performance in some tasks. A historical example is TD-Gammon: a program that achieved a level of play just slightly below that of the top human backgammon players of the time \cite{}. Another example that is more recent is AlphaGo: a program that is claimed to be arguably the strongest Go player in history \cite{}. In the future, an example might be self-driving cars. As computing power increases and new algorithms continue to be discovered \cite{}, reinforcement learning research will only grow in popularity, and so it is important to investigate it.

There is one drawback to existing reinforcement learning algorithms: they are most effective when there are no limits to the trial-and-error interactions an agent can make with its environment while learning. This is not an issue in virtual settings, but could cause problems in physical settings. For example, a computer chess program does not feel fatigue and is able to compute millions of games in minutes. On the other hand, an algorithm learning from scratch to drive a car will likely crash on its first run. For reasons such as not wanting to incur a repair cost or wasting time, it is preferable to be confident that a self-driving car has a reasonable ability to drive before testing it. To build confidence that a reinforcement learning algorithm will perform reasonably successfully in a real-world environment, we could first train the algorithm in a similar virtual environment, such as a simulation. Transfer learning is a key component of this process, since the algorithm's subsequent success depends critically on its ability to transfer the knowledge gained from the virtual domain and apply it to the real domain. For reasons such as this, transfer learning has become a crucial technique to build better reinforcement learning systems \cite{}. 

\subsection{Aims and achievements}
\subsubsection{Aims}
The aim of the project is to develop a working reinforcement learning system that can play one or more Atari games successfully, and can apply the knowledge gained to learn to play another different Atari game. For example, the system might be composed of one agent which has already learned to play Breakout, and another agent which is about to learn to play Pong; the design of the system lets the knowledge from the Breakout agent be applied by the Pong agent to make its own learning more efficient. Another example system might be composed of three agents which have already learned to play each of Breakout, Pacman, and Space Invaders respectively, and another agent learning Pong as before; again, the system lets the knowledge from the Breakout agent, Pacman agent, and Space Invaders agent be applied by the Pong agent to make its own learning more efficient. We refer to an agent applying knowledge gained by other agents as \textit{pre-training}. The question arises: \textit{What is the benefit of pre-training?}. More specifically, \textit{How much faster does an agent with pre-training learn a new game than one without? Twice as fast? Three times? Or not at all?}, and, \textit{Is pre-training on more games more beneficial?}. 

To address the aim, the objectives for the project were divided into three categories: minimum, intermediate, and advanced. 

The minimum objective involves developing a reinforcement learning algorithm that learns to perform successfully in its environment. A suitable problem to start with is balancing a pole on a cart (virtually). The algorithm should then be adapted to have agents learn to maximise score in Atari games.

The intermediate objective involves transfer learning by reusing weights in the network. The reinforcement learning algorithm should train two good agents for two Atari games. A third agent should then be trained on one game, with its network weights initialised to that of the agent trained on the other game. The third agent's performance should be compared with the agent trained on the same game to assess the benefit of pre-training.

The advanced objective involves transfer learning by training on the behaviour of one or more trained agents. We should train $n$ agents for $n$ Atari games, and generate 10,000 trajectories of 1,000 steps each from the trained agent for each game. We should then build a generative model and fit it to the trajectories produced by $n-1$ of the games, before transferring the model to the $n$th game. Similarly, the generative model's performance should be compared with the agent trained on the same game to assess the benefit of pre-training.

To quantify the benefit of pre-training, we propose two research questions: \textit{How large does the model need to be for the pre-training to be useful?}, and, \textit{How does the size of the effect change when the amount of data is reduced by 10x? By 100x?}.

\subsubsection{Achievements}
We achieved the minimum and intermediate objectives of the project. We developed a minimal PPO algorithm that learned to perform successfully in the CartPole environment. By introducing a convolutional neural network (CNN) to the policy and value networks, the algorithm was able to learn to perform successfully in the environments Pong and Breakout. By reusing the weights in the network of one of the agents, another agent was trained and its performance analysed. We observed there to indeed be a benefit of pre-training. The greater the amount of data, the greater the benefit there was, as measured by three metrics: jump-start performance, accumulated rewards, and final performance. The advanced objective was not able to be completed due to the challenges brought about by external limiting factors, such as Covid-19. 

\newpage
\section{Related work (2 pages)}
\label{section:related-work}
There have been several academic achievements in developing reinforcement learning algorithms that address challenging sequential decision-making problems, with  potential for real-world applications. For an agent to learn a good behaviour, the agent has to make decisions in an environment to optimise a given notion of cumulative rewards. We focus on existing solutions that achieve this by value-based methods or policy gradient methods. Other approaches include those that combine the two methods, or are model-based.

\subsection{Value-based methods}
One of the simplest and most popular value-based algorithms to define a policy is Q-learning. The basic version of Q-learning keeps a lookup table of values with one entry for every state-action pair \cite{DBLP:journals/corr/abs-1811-12560}. In order to learn the optimal Q-value function, the Q-learning algorithm makes use of the Bellman equation for the Q-value function, which has a unique solution. This idea was expanded to involve approximations of the Q-values \cite{10.5555/2998828.2998976}, and Q-values parameterised with a neural network where the parameters are updated \cite{10.1007/11564096_32}. More recent approaches use deep neural networks as function approximators \cite{Mnih2015}. Traditionally, the optimal policy was found by finding an exact solution; whereas, modern approaches approximate the optimal policy by using states as input to neural networks. 

One limitation of DQN-based approaches is that these types of algorithms are not well-suited to deal with large action spaces. In the Atari game Gravitar which has 18 actions, a DQN agent achieves a mean score of 306.7 (compared to 2672 by a human) \cite{Mnih2015}. 

\subsection{Policy gradient methods}
Policy gradient methods optimise a performance objective (typically the expected cumulative reward) by finding a good policy thanks to variants of stochastic gradient ascent with respect to the policy parameters \cite{DBLP:journals/corr/abs-1811-12560}. One common approach is to use an actor-critic architecture that consists of two parts: an actor and a critic \cite{article}. The actor
refers to the policy and the critic to the estimate of a value function (e.g.,
the Q-value function).

\subsection{Transfer learning}

\begin{itemize}
    \item This relates to my research question because \dots 
    \item Motivate why did things I did \dots
\end{itemize}

\newpage
\section{Solution (4-7 pages)}
In order to investigate the usefulness of pre-training for reinforcement learning algorithms , a specific workflow would need to be implemented. Firstly, a reinforcement learning algorithm would be used to train several agents in different environments. Secondly, trajectories of the trained agents would be generated. Thirdly, a generative model would be used to train on these trajectories. Finally, the generative model would be fine-tuned on the testing environment. Instead of training on Atari games straightaway, we built a model to train on cart-pole. This enabled the production of an intermediate solution to our problem. The downside of this solution is that it cannot work with complex problems requiring visual inputs. 

[Bridging paragraph] Rather than implement everything from scratch, build on frameworks to allow focus on algorithm design \dots

The solution is to implement the PPO algorithm and test it on OpenAI Gym. Then, the network weights are saved and applied to a different environment. 

\subsection{Specification and design}
\subsubsection{Specification}
First, the solution must be able to load a virtual environment. 

Second, the solution must be able to interact with the environment, by taking actions and making observations. The solution should indicate the actions available to be taken by an agent. When the agent takes an action, the environment should change, and the new state of the environment should be able to be observed.  

Third, the solution should indicate the reward achieved by the previous action.

Fourth, the solution should indicate when the environment needs resetting. 

\subsubsection{Design}
The design of the proximal policy optimisation algorithm is as follows: \dots 

The architectural diagram for this is /dots

\subsection{Implementation issues}
\begin{itemize}
    \item The features of the implementation process were \dots
\end{itemize}

\subsection{Tools used}
\begin{itemize}
    \item The tools used were \dots
    \item The algorithms used were \dots
\end{itemize}

\subsection{Verification and validation}
\begin{itemize}
    \item Verification was done by \dots 
    \begin{itemize}
        \item Do implementations work there? \dots 
        \item What do I do to judge the outcome/success?
        \item Try to answer whether transfer learning is generalisable 
    \end{itemize}
    \item Validation was done by \dots
\end{itemize}

\subsection{Testing}
[Bridging paragraph]
\begin{itemize}
    \item Testing was done by \dots
    \begin{itemize}
        \item reproduce on simple problems
    \end{itemize}
\end{itemize}

\newpage
\section{Results (2-3 pages)}
[Bridging paragraph]
\subsection{Evaluation method}
\begin{itemize}
    \item The evaluation methods adopted were \dots
\end{itemize}

\subsection{Experimental settings}
\begin{itemize}
    \item These were the experimental settings for each experiment carried out: \dots
\end{itemize}

\subsection{Results}
\begin{itemize}
    \item The results generated by the software were \dots
\end{itemize}

\newpage
\section{Evaluation (1-2 pages)}
[Bridging paragraph]
\subsection{Suitability of the approach (more SE, maybe exclude?)}
\begin{itemize}
    \item The approach was/was not suitable because \dots
    \item Was it a good idea to use PyTorch, etc.?
\end{itemize}

\subsection{Strengths and limitations of the algorithm} 
\begin{itemize}
    \item The strengths of the algorithm were \dots
    \item The limitations of the algorithm were \dots
    \item The lessons learnt were \dots
\end{itemize}

\subsection{Project organisation}
\begin{itemize}
    \item The project was organised as well as you would expect in a global pandemic \dots
\end{itemize}

\section{Conclusions (1 page)}
\subsection{Project overview}
\begin{itemize}
    \item The project was to \dots
\end{itemize}

\subsection{Main findings}
\begin{itemize}
    \item The main findings were as follows: \dots
    \item The conclusions from these findings were \dots
\end{itemize}

\subsection{Further work}
\begin{itemize}
    \item The project can be extended by \dots
\end{itemize}

\cite{}
\bibliography{references}
\end{document}