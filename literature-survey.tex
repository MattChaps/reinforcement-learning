\documentclass{article}

\usepackage{biblatex}
\usepackage{graphicx}
\usepackage{amssymb,amsmath}
\providecommand{\tightlist}{%
  \setlength{\itemsep}{0pt}\setlength{\parskip}{0pt}}

\addbibresource{references.bib}

\title{Reinforcement Learning: A Literature Survey}
\author{Matthew Chapman}

\begin{document}
\maketitle

\section{Introduction}
Reinforcement learning is the class of problems concerned with an agent learning behaviour through trial-and-error interactions with a dynamic environment \cite{Kaelbling1996}.

An example is an aspiring tightrope artist (the agent) learns to walk
from one end of a rope to another without falling (the behaviour) by
repeatedly correcting their balance (the trial-and-error interactions) whenever the rope wobbles beneath them (the dynamic environment).

In this literature survey we: 
\begin{enumerate}
  \item{describe the problems we want to solve with reinforcement learning, and explain why they are interesting to solve;} 
  \item{describe historical and current work in the field, including the kinds of problems solved and the approaches; as well as,} 
  \item{state what we will be using in our own project.}
\end{enumerate}

Much of the environments mentioned in this literature survey are of games, as they (board games in particular) are ideal testing grounds for exploring concepts and approaches in reinforcement learning and artificial intelligence \cite{Tesauro1995}.

\section{Motivation}
Reinforcement learning is an area of machine learning, which in turn is a subset of artificial intelligence. A major goal of some research is to develop artificial general intelligence (AGI), a machine capable of understanding or learning any intellectual task that a human can. Solving the AGI problem is interesting, as it would prove that intelligence can be programmed without divine intervention. 

Reinforcement learning agents exhibit behaviour similar to intelligence, with intelligence defined as ``adaptation with insufficient knowledge and resources'' \cite{OnDefiningArtificialIntelligence}. It is also said that humans and other animals use reinforcement learning to understand their environment and generalise past experience to new situations \cite{Mnih2015}. To study reinforcement learning, therefore, is to study a small area of exhibiting human intelligence in machines, and solving reinforcement learning problems, such as those done in robotics \cite{Kober2013} or chip design  \cite{Ipek2008}, can lead to developments that improve quality of life and ultimately contribute to the search for AGI.

\section{Markov Decision Processes}

The classical framework used to abstract a real problem as a mathematically idealised reinforcement learning problem is the Markov Decision Process (MDP).

Together, the agent and MDP give rise to a trajectory like this:

\begin{enumerate}
  \item{At time $t$, the agent senses the state of its environment, $S_t$.}
  \item{The agent selects an action, $A_t$, which affects the environment.}
  \item{The environment is changed, now at time $t+1$, and returns a reward signal, $R_{t+1}$, to the agent, and the process is repeated.}
\end{enumerate}

\begin{center}
  \includegraphics[width=1.0\textwidth]{images/fig-3.1.png}
\end{center}

The agent selects actions dictated by its policy, $\pi$, and the goal of the agent is to maximise the total reward it receives from the environment over time.

In order to maximise the total reward over time, the agent must consider more than just immediate rewards.

The \textbf{return} $G_t$ is defined as the discounted total future reward, 

\[ 
G_t = R_{t+1} + \gamma R_{t+2} + \gamma^2 R_{t+3} + \dots = \sum_{k=0}^{\infty} \gamma^k R_{t+l+1},
\]

where $\gamma$ determines the weight (or importance) given to rewards at later time-steps.

The \textbf{value function of a state $s$ under a policy $\pi$} is the total reward an agent can expect to accumulate, starting from $s$ and following $\pi$ thereafter:

\[
\begin{aligned}
v_\pi(s) \doteq \mathbb{E}_\pi \left[ G_T \middle| S_t = s \right].
\end{aligned}
\]

Every state must satisfy this condition.

The \textbf{value function of taking action \(a\) in state \(s\) under a policy \(\pi\)},

\[
\begin{aligned}
q_\pi(s, a) & \doteq \mathbb{E}_\pi[G_t | S_t = s, A_t = a] \\
& = \mathbb{E} \left[ \sum_{k=0}^{\infty}\gamma^k R_{t+k+1} \middle| S_t = s, A_t = a \right],
\end{aligned}
\]

is the expected return starting from \(s\), taking the action \(a\), and
thereafter following policy \(\pi\).

Solving the \textbf{Bellman optimality equation for \(q_*\)},

\[\begin{aligned} 
q_*(s,a) & = \mathbb{E} \left[ R_{t+1} + \gamma \max_{a'} q_*(S_{t+1},a') \middle| S_t=s, A_t=a \right], 
\end{aligned}\]

provides one route to finding an optimal policy, and thus to solving the reinforcement learning problem. However, the solution relies on at
least three assumptions that are rarely true in practice:

\begin{enumerate}
\def\labelenumi{\arabic{enumi}.}
\item
  the dynamics of the environment are accurately known;
\item
  computational resources are sufficient to complete the calculation; and,
\item
  the states have the Markov property.
\end{enumerate}

Section 5 highlights methods that circumvent this.

\section{Dynamic Programming}

Dynamic programming (DP) refers to the collection of algorithms
that can be used to compute optimal policies given a perfect model of
the environment as a Markov decision process (MDP).

We usually assume that the environment is a finite MDP. A common way of
obtaining approximate solutions for tasks with continuous state and
actions is to quantise the state and action spaces and then apply
finite-state DP methods.

The key idea of DP, and of reinforcement learning generally, is the use
of value functions to organise and structure the search for good
policies.

How do you compute the state-value function \(v_\pi\) for an arbitrary policy \(\pi\)? If the environment's dynamics are completely known, then iterative solution methods are most suitable.

\textbf{Iterative policy evaluation:}

\begin{enumerate}
\def\labelenumi{\arabic{enumi}.}
\item
  Initial approximation, \(v_0\), is chosen arbitrarily.
\item
  Each successive approximation is obtained by using the Bellman
  equation for \(v_\pi\) as an update rule.
\end{enumerate}

\textbf{Policy improvement:} the process of making a new policy that
improves on an original policy, by making it greedy with respect to the
value function of the original policy

\textbf{Value iteration}

\[
\begin{aligned}
  v_{k+1}(s)
  & \doteq \max_a \mathbb{E} \left[ R_{t+1} + \gamma v_k(S_{t+1}) \middle| S_t = s, A_t = a \right] 
  \\
  & = \max_a \sum_{s', r} p \left( s', r \middle| s, a \right) \left[ r + \gamma v_k(s') \right], 
  \quad \text{ for all $s \in \mathcal{S}$}
\end{aligned}
\]

DP methods involve operations over the entire state set of the MDP. If
the state set is very large, like in Backgammon, then each set is
expensive.

\emph{Asynchronous} DP algorithms update the value of states in any
order whatsoever, using whatever values of other states happen to be
available.

Policy iteration consists of making the value function consistent with
the current policy (policy evaluation), and the other making the policy
greedy with respect to the current value function (policy improvement).
The result is convergence to the optimal value function and an optimal
policy.

\section{Monte Carlo Methods}

To estimate the value of a state from experience is to
average the returns observed after visits to that state. As more returns
are observed, the average should converge to the expected value. This is the basis of Monte Carlo (MC) methods for learning the state-value function for a given policy.

Defining a visit to \(s\) as each occurrence of state \(s\) in an episode, there are two main MC methods:

\textbf{First-visit MC method:} estimates \(v_\pi(s)\) as the average of
the returns following first visits to s.

\textbf{Every-visit MC method:} averages the returns following all
visits to \(s\).

Both methods converge to \(v_\pi(s)\) as the number of visits (or first
visits) to \(s\) goes to infinity.

There are three advantages of Monte Carlo methods over DP methods:

\begin{itemize}
\tightlist
\item
  The computational expense of estimating the value of a single state is
  independent of the number of states;
\item
  The ability to learn from actual experience; and,
\item
  The ability to learn from simulated experience.
\end{itemize}

If a model is not available, then it is particularly useful to estimate
\emph{action} values rather than \emph{state} values. With a model,
state values alone are sufficient to determine a policy. One of our
primary goals for Monte Carlo methods is to estimate \(q_*\). To achieve
this, we first consider the policy evaluation problem for action values.

The policy evaluation problem for action values is to estimate
\(q_\pi(s,a)\). The MC methods for this are essentially the same as just
presented for state values, replacing state with state-action pair.

The complication is that many state-action pairs may never be visited.
To compare alternatives we need to estimate the value of \emph{all} the
actions from each state. One way to do this is by specifying that the episodes \emph{start in a state-action pair}, and that every pair has a nonzero probability of being selected at the start.

Monte Carlo methods can be used to find optimal policies given only
sample episodes and no other knowledge of the environment's dynamics.

Since the assumption of exploring starts is unlikely, the only general way to ensure that all actions are selected infinitely often is for the agent to continue to select them. There are two approaches:

\textbf{On-policy methods:} evaluate or improve the policy that is used
to make decisions.

\textbf{Off-policy methods} evaluate or improve a policy different from
that used to generate the data.

All learning control methods face a dilemma: they seek to learn action
values conditional on subsequent \emph{optimal} behaviour, but they need
to behave non-optimally in order to explore all actions (to \emph{find}
the optimal actions). The on-policy approach is a compromise --- it
learns action values not for the optimal policy, but for a near-optimal
policy that still explores. A more straightforward approach is to use
two policies, one that is learned about and that becomes the optimal
policy, and one that is more exploratory and is used to generate
behaviour.

\textbf{Target policy:} the policy being learned

\textbf{Behaviour policy:} the policy used to generate behaviour

Learning is from data ``off'' the target policy, and the overall process
is termed \emph{off-policy} learning.

Off-policy methods utilise \emph{importance sampling}

\textbf{Importance sampling:} technique for estimating expected values
under one distribution given samples from another.

Apply importance sampling to off-policy learning by weighting returns
according to the relative probability of their trajectories occurring
under the target and behaviour policies, called the
\emph{importance-sampling ratio}.

The importance-sampling ratio depends only on the two policies and the
sequence, not on the MDP.

There are two types of importance sampling: ordinary, and weighted.

Ordinary importance sampling is unbiased whereas weighted importance
sampling is biased (though the bias converges asymptotically to zero).

Monte Carlo prediction methods can be implemented incrementally, on an
episode-by-episode basis. In Monte Carlo methods we average
\emph{returns}.

An advantage is that the target policy may be greedy, while the
behaviour policy can continue to sample all possible actions.

These methods follow the behaviour policy while learning about and
improving the target policy.

\section{Deep Q-networks}
Q-learning is a value-based class of algorithms that aim to build a value function, which subsequently lets us define a policy.

$Q$-learning keeps a lookup table of values $Q(s,a)$ with one entry for every state-action pair. In order to learn the optimal $Q$-value function, the $Q$-learning algorithm makes use of the Bellman equation for the $Q$-value function (Bellman and Dreyfus,1962) whose unique solution is $Q^*(s,a)$.

This is often inapplicable with a high-dimensional state-action space.

Deep Q-networks (DQN) combine reinforcement learning with deep neural networks. Mnih et al. \cite{Mnih2015} used a deep convolutional neural network to approximate the optimal action-value function. To address instability in learning when a nonlinear function approximator such as a neural network is used, a mechanism called experience replay was employed with an iterative update that adjusts the action-values (Q) towards target values that are only periodically updated, thereby reducing correlations with the target. The agent they trained achieved a level comparable to that of a professional human games tester across a set of 49 ATARI games.

\section{Our Project}

In this project, we want to look at deep reinforcement learning algorithms, implement them, and analyse their performance in those environments provided by OpenAI Gym.

\printbibliography
\end{document}